%%%%%%%%%%%% DIRECIONAMIENTO ORGANIZACIÓN A  %%%%%%%%%%%%
	
\section{Organización A.}

\par La red de la organización A tiene asignado el rango de direcciones IP \textbf{173.89.0.0/22}, siendo 22 el número de bits de la máscara de red. Todas las subredes a definir en dicha red, así como los equipos que pertenezcan a ellas tienen también que compartir esos 22 bits más significativos.
\par El primer paso es definir cada una de las \textbf{subredes} dentro de la organización:
\begin{itemize}
	\item \emph{\textbf{4 subredes de área local}}: LAN1.0, LAN1.1, LAN1.2  y LAN1.3
	\item \emph{\textbf{7 subredes P2P}}: P2P1.0 (RouterA0-A1), P2P1.1 (RouterA1-A2), P2P1.2 (RouterA0-A3), P2P1.3 (RouterA1-A3), P2P1.4 (RouterA2-A3), P2P1.5 (RouterA3-A4) y P2P1.6 (RouterA2-B0).
\end{itemize}
\par Para determinar el número mínimo de direcciones IP de una subred hay que tener en cuenta que se necesita una dirección IP para cada equipo conectado a la misma y una dirección IP para la interfaz de los routers que den acceso a esa subred. Además, se necesitan 2 direcciones IP adicionales: la reservada para la dirección de subred y la dirección broadcast.
\par En la tabla \ref{tab:orgA.1} aparecen las distintas subredes junto al número de interfaces (host e interfaces de los routers) que lo conforman, y el número de bits que forman el SubNetID y el HostID.
%%%% TABLA DE SUBREDES DE LA ORGANIZACIÓN A %%%%
\definecolor{azul}{rgb}{0.36, 0.54, 0.66}
\definecolor{naranja}{rgb}{1.0, 0.56, 0.0}
\definecolor{lila}{rgb}{0.62, 0.0, 0.77}
\definecolor{rosa}{rgb}{0.89, 0.31, 0.61}
\definecolor{verde}{rgb}{0.3, 0.73, 0.09}
\begin{table}[H]
    \centering
    \begin{tabular}{|c|c|c|c|}
        \hline
       	Subred&Número de Interfaces&Bits SubNetID&Bits HostID\\\hline
	\cellcolor{azul}{LAN1.0}&509&1&9 \\ \hline
	\cellcolor{naranja}{LAN1.1}&55&4&6 \\ \hline
	\cellcolor{lila}{LAN1.2}&250&2&8 \\ \hline
	\cellcolor{rosa}{LAN1.3}&125&3&7 \\ \hline
	\cellcolor{verde}{P2P1.0}&2&9&2 \\ \hline
	\cellcolor{verde}{P2P1.1}&2&9&2 \\ \hline
	\cellcolor{verde}{P2P1.2}&2&9&2 \\ \hline
	\cellcolor{verde}{P2P1.3}&2&9&2 \\ \hline
	\cellcolor{verde}{P2P1.4}&2&9&2 \\ \hline
	\cellcolor{verde}{P2P1.5}&2&9&2 \\ \hline
	\cellcolor{verde}{P2P1.6}&2&9&2 \\ \hline
    \end{tabular}
    \caption{Tabla de subredes de la organización A.}
    \label{tab:orgA.1}
\end{table}

