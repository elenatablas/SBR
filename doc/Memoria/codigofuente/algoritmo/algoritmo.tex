%%%%%%%%%%%%  Algoritmo  %%%%%%%%%%%%

\begin{center}
	{\fboxrule=4pt \fbox{\fboxrule=1pt
		\fbox{\LARGE{\bfseries 2. Diseño del algoritmo del MI}}}} \\
	\addcontentsline{toc}{chapter}{2. Diseño del algoritmo del Motor de Inferencias}
	\setcounter{chapter}{2}
	\setcounter{section}{0}
	\rule{15cm}{0pt} \\
\end{center}

\section{Pseudocódigo}
%%% CÓDIGO %%%
\begin{listing}[language=Pascal]
function ENCAMINAMIENTO-HACIA-ATRAS
  BH=HechosIniciales;
  si Verificar(Meta,BH) entonces 
	Imprimir(Meta, FactorCerteza);
 	devolver "exito";
  si no
    devolver "fracaso";
  fin si
\end{listing}
\begin{listing}[language=Pascal]
function VERIFICAR
  Verificado=Falso;
  si Contenida(Meta,BH) entonces devolver "Verdadero";
  si no
    ConjuntoConflicto=Equiparar(Consecuentes(BC),Meta);
    mientras NoVacio(ConjuntoConflicto) hacer
      R=Resolver(ConjuntoConflicto);
	  Eliminar(R,ConjuntoConflicto);
	  NuevasMetas=ExtraerAntecedentes(R);
	  Verificado=Verdadero;
	  mientras NoVacio(NuevasMetas) y Verificado hacer
	  	Nmet=SeleccionarMeta(NuevasMetas);
		Eliminar(NMet, NuevasMetas);
		Verificado=Verificar(NMet,BH);
	  fin mientras
	  si noEsTipoLiteral(R) entonces Caso1(R,BH);
	  fin si
	  Caso3(R,BH);
	  Insertar(ConjuntoResuelto,R); 
	fin mientras
	// Caso2(Meta,ConjuntoResuelto);
	InsertarObjetivo(Meta,BH,ConjuntoResuelto); 
  fin si
  devolver(Verificado);
\end{listing}
\newpage
\par El pseudocódigo es parecido al proporcionado por la asignatura. \\
Elimino ``\texttt{No(Verificado)}'' del bucle exterior porque quiero recorrer todas
las reglas posibles para generar el factor de certeza de mi objetivo. Además,
añado el ``\texttt{ConjuntoResuelto}'' para poder calcular los factores de certeza 
cuando se convinan varias reglas que es el caso de la adquisición incremental de evidencia.
\par Los factores de certeza se calculan en distintos instantes. 
Cuando ya he recorriendo las NuevasMetas que son los antecedentes de la regla que he seleccionado 
y he verificado todos los antecedentes que la componen, observo si solo tiene un antecedente o
de varios. En el segundo caso, combinaré sus factores de certeza para obtener el de la conjunción o disyunción. 
A continuación, para todos los casos, combinaré la evidencia con la regla.
Por último, cuando ya he calculado todos los factores de certeza de las evidencias con las reglas que tienen como consecuente la meta
que se está buscando, combino todas esas reglas para formar un único factor de certeza de la meta. \\





