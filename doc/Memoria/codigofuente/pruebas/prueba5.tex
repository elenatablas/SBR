\section{Prueba 5}
\subsection{Enunciado del problema}
Estamos en pandemia, los poblacion sale de fiesta con 0.5, a veces lleva la mascarilla con certeza de 0.4, utiliza el transporte público con 0.3 y no suele pasear (entendemos que su factor de certeza es de 0.2).
\par Si los casos de covid aumentan, la evidencia de que nos confinen es de 0.5. Si usamos el transporte público o paseamos es una evidencia a favor (0.4) de que hay más casos.
En cambio, Si usamos el transporte público y salimos de fiesta es una evidencia a favor (0.9) de que hay más casos.
Además, sabemos que cuando la población no lleva mascarilla, se tiene una evidencia de 0.8 los casos de covid aumentan. Si la población sale de fiesta  y hay casos covid entonces se reduce el aforo (entendemos que su factor de certeza es 0.3). 
También, si la población no lleva mascarilla, se reduce siempre el aforo en ciertos lugares. Si se reduce el aforo es una evidencia de 0.4 que nos confinen.
¿La población se confina?
\subsection{Formalización del problema}
\par Sea la siguiente signatura:
\par \texttt{$\sum$ = \{casos, noMascarilla, aforo, fiesta, transporte, pasear, confinar\} }
\par donde
\begin{itemize}
    \item \texttt{casos} = ``Los casos de covid aumentan.''
    \item \texttt{noMascarilla} = ``La población no lleva mascarilla.''
    \item \texttt{aforo} = ``El aforo se reduce en ciertos lugares.''
    \item \texttt{fiesta} = ``La población sale de fiesta.''
    \item \texttt{transporte} = ``La población viaja en transporte público.''
    \item \texttt{pasear} = ``La población pasea.''
    \item \texttt{confinar} = ``La población se confina.''
\end{itemize}

\subsection{Base de Conocimiento (BC)}
Reglas:
\begin{itemize}
    \item R1: Si \texttt{casos} Entonces \texttt{confinar}, $FC=0.5$
    \item R2: Si \texttt{transporte} o \texttt{pasear}, Entonces \texttt{casos}, $FC=0.4$
    \item R3: Si \texttt{transporte} y \texttt{fiesta}, Entonces \texttt{casos}, $FC=0.9$
    \item R4: Si \texttt{noMascarilla} Entonces \texttt{casos}, $FC=0.8$
    \item R5: Si \texttt{fiesta} y \texttt{casos} Entonces \texttt{aforo}, $FC=0.8$
    \item R6: Si \texttt{noMascarilla} Entonces \texttt{aforo}, $FC=1$
    \item R7: Si \texttt{aforo} Entonces \texttt{confinar}, $FC=0.4$
\end{itemize}

\subsection{Base de Hechos (BH)}
Hechos:
\begin{itemize}
    \item \texttt{fiesta}, $FC(\texttt{fiesta}) = 0.5$
    \item \texttt{transporte}, $FC(\texttt{transporte}) = 0.3$
    \item \texttt{pasear}, $FC(\texttt{pasear}) = 0.2$
    \item \texttt{noMascarilla}, $FC(\texttt{noMascarilla}) = 0.4$
\end{itemize}
\newpage
