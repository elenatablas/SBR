 %%%%%%%%%%%%% BIBLIOGRAFÍA %%%%%%%%%%%%%%%%
 \begin{center}
	{\fboxrule=4pt \fbox{\fboxrule=1pt
		\fbox{\LARGE{\bfseries 6. Referencias bibliográficas}}}} \\
	\addcontentsline{toc}{chapter}{6. Referencias bibliográficas}
	\rule{15cm}{0pt} \\
\end{center}
\par Para la elaboración de este proyecto de prácticas he hecho uso del material disponible en la página de la asignatura en el aula virtual que incluye las transparencias de teoría y de prácticas, junto con los ejercicios resueltos relacionados con estas prácticas.
\par Por último, he consultado las siguientes páginas de la web de \texttt{Cisco} por complementar información:
\begin{enumerate}
	\item Cisco. (28 de Enero de 2008). Sample Configuration of Triggered Extensions to RIP. Recuperado de 
	\raggedright\url{https://www.cisco.com/c/en/us/support/docs/ip/routing-information-protocol-rip/13720-51.html}
	\item Cisco. Networking Academy. Cisco Packet Tracer. Recuperado de 
	\url{https://www.netacad.com/es/courses/packet-tracer}
\end{enumerate}

